%************************************************
\chapter{Discussion}\label{ch:Discussion} 
%************************************************
\section{Suitability of SID}
The Shared Imbalance Detection (SID)) tool is capable of taking a feature extraction file and by applying a Z score algorithm can detect regions of shared CNV from manually created true positive cases with a sensitivity of 100\% in a timely fashion.

\paragraph*{}
SID meets the functional 

\section{Z score cutoff to define a probe as abnormal}
Analysis of the test cases suggest that using a Z score of between 3.55 and 4 to define a probe as abnormal would detect shared copy number variation without any false positive calls. However applying this threshold to prospective arrays, the same arrays used to create the test cases, resulted in calls in 1 in 10 cases. 

\section{Benign CNVs}
It is expected that hybridisation partners would share one or more of many known, common benign CNVs. SID would correctly identify these regions as abnormal, however the above analysis would classify these calls as false positives.
\paragraph*{}
Should SID be adopted by the clinical service calls within known benign CNVs could be filtered out, or the experience of clinical scientists would make analysis of common calls trivial.
\paragraph*{}
It may be worth excluding these regions from future specificity and sensitivity calculations however no regions should be filtered when applied to live arrays.

\section{Calls smaller than 5 calls}
The vast majority of false positive calls contain less than 5 probes (table X showing breakdown of test cases and table of calls in prospective arrays). The average signal intensities of probes within a true positive call is higher than that of a false positive call (link to fig) so a secondary analysis, could be applied to further reduce false positive calls. 


\section{Different cutoffs for gain/loss}
The proportional signal intensity change between diploid and triploid and diploid and monosomy is the same ($\pm$50\%). Therefore the Z score would be the same for a loss and a gain. 
Similar numbers of false positive gains and losses (fig) suggest using the same cutoff for losses and gains does not introduce bias however should further work identify a lack of sensitivity for gain or loss separate cutoffs could be applied.

\section{Alternative approaches to consecutive probes analysis}
\subsection{Larger sliding windows}
The Z score algorithm implemented differs to that suggested by Agilent \cite{agilenttechnologies2012}. Agilent suggest a sliding window and the proportion of probes outside a cutoff used to define a region as abnormal.

In order to meet the functional specification of detection of three consecutive probes the sliding window must be very small, and the proportion of abnormal probes within a region would be very high. This may however result in a call which does not meet the specification (Figure \ref{fig:incorrectcallandalternatecomb}a)

\begin{figure}[h]
\centering
\includegraphics[width=1\linewidth]{./Figures/incorrectcallandalternatecomb}
\caption{a. b.}
\label{fig:incorrectcallandalternatecomb}
\end{figure}
\paragraph*{}
This approach would however prevent aberrations such as those in Figure \ref{fig:incorrectcallandalternatecomb}b being missed.
\paragraph*{}
The consecutive probe approach is designed to match the routine analysis approach however a sliding window approach may be worth investigating. Small overlapping windows can be applied and combined similar to the existing tiling strategy.

\subsection{Sliding window - Regions of interest}
A variation of the sliding window approach was investigated, using defined regions of interest (every CNV which has been reported more than twice) as a window. The same Z score analysis was performed and the proportion of abnormal probes within the region used to determine if a region is abnormal or not.
\paragraph*{}
Despite this approach being targeted to known abnormalities and the potential to expand the list of regions this approach is more limited than the consecutive probe approach so was not pursued.

\section{Further work}
\subsection{Sex Chromosomes}
Further work is required to include sex chromosomes in the analysis. The gender of the sample affects the signal intensities of probes on the sex chromosomes. To identify CNV on the sex chromosomes a gender specific reference range is required. The average signal intensities across the X and Y chromosomes could determine gender and therefore which reference range should be applied.
\subsection{Sample Type}
This project has only assessed postnatal blood samples. The array service also processes prenatal, solid tissue and preimplantation genetic diagnosis samples, which have variable resolution and performance due to the quality of the sample. 
\paragraph*{}
Differing analysis strategies and Z score cutoffs would be required for each sample type in line with existing analysis approaches in addition to a method for identifying the sample type of each array.
\subsection{What happens when shared CNV is detected?}
Once shared CNV is detected further investigations are required. The level of investigation required would dictate the acceptable levels of specificity of SID. If a thorough investigation is required the time and cost incurred would negate any savings brought by replacing phenotype mismatching with SID.
\paragraph*{}
The signal intensities can be examined within the analysis software which would confirm or deny the presence of CNV. This practise is in line with current analysis methods when identifying if an aberration is a gain in one hybridisation partner or a loss in the other. However having \textgreater 100 calls on an array would make this approach unfeasible.
\paragraph*{}
Another potential approach is EvE, a tool which allows post hoc hybridisation of samples. Each sample could be independently compared against another sample, such as a reference control, to see if CNV is present. 
\paragraph*{}
EvE takes about one minute to run and the analysis could be performed in less than 10 minutes. However should this be required routinely any time savings would be diminished.
EvE has been validated for use with the PGD service but has not yet been validated for use with postnatal blood samples.
\paragraph*{}
An extreme approach is to re-run the sample against different hybridisation partners. If this is the case the specificity must be well understood to avoid a large number of samples being repeated, removing any cost savings.
\paragraph*{}
The approach taken would require a thorough consultation with clinical scientists and bioinformaticians,

\subsection{Exporting results to LIMS}
The SID database will be a separate database to the departmental LIMS database, hosted on the same server.
\paragraph*{}
A function could therefore be added to the end of SID to connect to the LIMS system and export the results directly to the LIMS system. The filename can be used to link to the two patient records using an existing query.
\paragraph*{}
The LIMS system would require an extra table to hold the shared CNV. Clinical scientists should be consulted to determine where and how this information should be displayed.
\paragraph*{}
Confirmation that analysis has completed successfully should be included for samples with no shared CNV.

\subsection{Initiation of SID}
Array runs occur at scheduled times during the week. SID scans a directory and compares the files in the directory to those already processed (stored in the feparams table) which would enable the script to be scheduled to run at specific times.
\paragraph*{}
Alternatively the script could be set off manually by the array technician once the feature extraction process has completed. An SOP and training would be required.

\subsection{Migration of database to the server}
The database has not been migrated to the SQL server on the server.  Interacting across a network into a remote server may increase the time taken to process each sample and the stability of this connection tested.
\paragraph*{}
If run overnight an increase in time may be tolerable however a process to identify and remedy interrupted or failed processing will be required. 

\section{Long term management of SID}
\subsection{Unit test}
After any changes are made to the script or database a unit test, utilising a truth set from a subset of the test set, should be performed to ensure no loss of function.

\subsection{Recalculation of the reference range}
The measured signal intensity can be affected following any maintenance or change to the microarray scanner. Following any maintenance a new reference range should be calculated, possibly using the first array run following maintenance. 
\paragraph*{}
This reference range should be created in a new table and the SID python script updated to apply the new reference and record the reference range applied in the feparams table.

\subsection{Truncating the database to maintain performance}
Each array run consists of 48 or 96 arrays. As the run is processed and the Features table grows in size the time taken to process each array increases up to 7 fold.
\paragraph*{}
Approaches to control SQL transactions failed to improve this situation however should this processing time become prohibitive alternative insert approaches can be considered, such as creating and inserting the features from a \ac{csv} file. This approach proved slower when uploading single arrays.
\paragraph*{}
Truncation of the features table, following analysis of each array or array run would maintain performance.
\paragraph*{}
Alternatively a features table could be created for each run, however a table holding 96 samples would be around 800MB which, with at least two runs per week, would quickly become prohibitively large. Given the relative short processing time, high throughput of arrays and long term availability of feature extraction files if a re-analysis is required it would be sensible to re-run SID than to return to stored values.
\paragraph*{}
The probeorder, feparam and referencevalues tables should not be truncated. The feparam table holds the filenames used by the python script to prevent feature extraction files being reprocessed.  The referencevalues table enables re-analysis of a sample in the future after a new reference range has been created and the probeorder table records the array design.
\paragraph*{}
The consecutive\_probe\_analysis table could be truncated once the shared CNVs exported to the LIMS.

\subsection{Contingency plan}
Should SID be implemented and then become unavailable contingency plans include temporarily suspending the array service or reverting to the current strategy of manually phenotype mismatching hybridisation partners.
\subsection{Database backup}
SID will be incorporated to Trust IT`s existing database backup strategy, utilising a 3rd party software to perform daily backups, keeping the last three backups. 

\subsection{Validation}
ISO laboratory quality accreditation requires thorough validation of all tests and equipment. To validate SID 100 arrays can be analysed and 3 common CNVs assessed. The predictions made by SID can be compared to the signal intensities inspected manually on the array trace. The workload of manual assessing 300 regions is inline with validation of other tests within the department.

\subsection{Documentation}
Three documents are required to accompany SID:
\paragraph*{}
ISO accreditation requires a validation document which details the steps taken to confirm SID performs as expected..
\paragraph*{}
A SOP or user manual describes how SID is run and explaining some common error messages.
\paragraph*{}
A development manual would be aimed at members of the bioinformatics department who will be responsible for developing and maintaining the script and database.. This will describe in depth how SID works, dependencies and a change log. 
\\
This would also contain instructions to perform tasks which the user cannot perform, for example completely removing arrays to enable the array to be reprocessed and creation and adoption of new reference ranges.
\paragraph*{}
These documents would be stored in the controlled document management system (Q-Pulse) and on the laboratory wiki which hosts the current standard operating procedures for the microarray service.

\section{Future considerations}
\subsection{Array redesign/multiple array platforms}
The department currently has three different array designs, the routine design, one high resolution array used only for specific cases and a legacy design. 
\paragraph*{}
Should SID replace phenotype mismatching SID should also be required to process these array files. Each array design would require a probeorder table and a reference range. The array design is contained within the filename. SID would need updating to extract this to apply the correct tables.
\paragraph*{}
Similarly any future changes to the routine array design would require the same measures.