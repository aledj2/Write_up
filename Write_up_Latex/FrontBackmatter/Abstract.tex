%*******************************************************
% Abstract
%*******************************************************
%\renewcommand{\abstractname}{Abstract}
\pdfbookmark[1]{Abstract}{Abstract}
\begingroup
\let\clearpage\relax
\let\cleardoublepage\relax
\let\cleardoublepage\relax
\chapter*{Abstract}
\textbf{Introduction:} Array CGH is a well established technique to detect copy number variation (CNV) in diagnostic genetic laboratories. To reduce the cost of the array service a patient to patient hybridisation approach was adopted, halving the costs of consumables. This however, introduces additional challenges, as arrays are analysed by directly comparing hybridisation partners, if a CNV is present in both hybridisation partners relatively there is no change so the CNV is not detected. This is currently mediated by manually selecting hybridised partners based on phenotype, a time consuming process.
\\ The aim of this project is to investigate a tool which can detect CNV independent of the hybridisation partner, enabling shared CNV to be detected.
\\ 
\textbf{Methods:} The shared imbalance detection (SID) tool was created, consisting of a database and Python script, which performs a Z score analysis on the signal intensities of each probe. Consecutive probes scored as abnormal in both hybridisation partners are reported as shared CNV. Two true positive datasets were created to train and test the Z score threshold. 
\\
\textbf{Results:} A range of Z score thresholds were applied to define a probe as normal or abnormal. A call was made in the expected region in 94\% of cases in a training set of 81 arrays, however this dataset was found not to be an accurate representation of shared CNV. A more representative test set of 69 arrays had calls in the expected region in 100\% of cases with no false positive calls using a Z score threshold of 3.55. 121 unmodified arrays were analysed in a prospective study.
\\
\textbf{Discussion:} SID can detect shared CNV in line with the departmental analysis protocols and in a timely manner. Long term management and further work, including incorporating analysis of the sex chromosomes and validation and implementation the service is described.
\endgroup			